\chapter{Основная часть}

\section{Выбор способа реализации???}

Предположим, мы захотели добавить какую-то новую функциональность в язык. Есть два принципиально разных способа это сделать:
\begin{itemize}
	\item использовать существующие средства языка
	\item изменять синтаксис языка (например, добавить новые ключевые слова)
\end{itemize}

У обоих этих подходов есть как положительные, так и отрицательные стороны. \textit{расписать плюсы и минусы} ИтогеЖ выбираем первый способ. 

В случае Kotlin есть несколько способов добавить поддержку неизменяемости объектов в язык. В работе IGJ это сделано с помощью добавления дополнитьельного типового параметра ко всем классам, как это сделано в работе \textit{ссылка}. Но это выглядит очень громоздко и трудно читаемо. Более того, информация о типовых параметрах остуствует в скомпилированных файлах, то есть информация об изменяемости также будет доступна только в иходном коде. Другой варинат -- использование аннотаций. 

Аннотация в Java -- это вид синтаксических методанных, которые могут быть добавлены в исходный код. Они могут быть доступны на этапе компилляции, встроены в класс-файлы, а также могут использоваться JVM во время исполнения программы. В Java аннотации можно применять к пакетам, классам, методам, переменным и параметрам. 

\section{Система аннотаций}

\section{Алгоритм вывода аннотаций}

\section{Сравнение с существующими подходами}


