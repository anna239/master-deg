\chapter{Основная часть}

\section{Подход к технической реализации}

Предположим, нужно добавить какую-то новую функциональность в язык. Есть два принципиально разных способа это сделать:
\begin{itemize}
	\item использовать существующие средства языка
	\item изменять синтаксис языка (например, добавить новые ключевые слова)
\end{itemize}

У обоих этих подходов есть как положительные, так и отрицательные стороны. Изменение синтаксиса языка приводит к изменению грамматики языка, из чего следует невозможность использования многих существующих инструментов для разработки с использованием этого языка, таких как компиляторы, среды разработки, различные анализаторы кода. Но с другой стороны этот подход позволяет добавлять в язык развитую систему выразительных средств. Использование же существующих средств языка ограничивает свободу введения новых концепций, но этот подход обычно гораздо проще в реализации.

В случае Java есть несколько способов добавить поддержку неизменяемости объектов в язык. В работе IGJ это сделано с помощью добавления дополнитьельного типового параметра ко всем классам. Но это выглядит очень громоздко и трудно читаемо. Более того, информация о типовых параметрах остуствует в скомпилированных файлах, то есть информация об изменяемости также будет доступна только в исходном коде. Другой варинат -- использование аннотаций. 

Аннотация в Java -- это вид синтаксических методанных, которые могут быть добавлены в исходный код. Они могут быть доступны на этапе компилляции, встроены в класс-файлы, а также могут использоваться JVM во время исполнения программы. В Java 7 аннотации можно применять к пакетам, классам, методам, переменным и параметрам. 

Как справедливо отмечают некоторые авторы, аннотации в том виде, в котором они реализованы в Java 7, не достаточно мощны для того, чтобы добавить поддержку контроля за изменяемостью объектов, так как в нынешней реализации нельзя аннотировать типы. Но уже в Java 8 такая поддержка появится, поэтому в данной работе именно аннотации используются для выражения неизменяемости объектов.

\section{Система аннотаций}

В данной работе каждая ссылка имеет модификатор изменяемости, который определяет, может ли быть изменено ее абстрактное состояние. Этот моификатор определяется на уровне исходного кода, анализуруется на этапе компиляции и может иметь одно из четырех значений: Mutable, Immutable, ReadOnly или Isolated. 

\subsection{Ссылочная неизменяемость}

Для поддержки ссылочной неизменяемости достаточно двух модификаторов: Mutable и ReadOnly. Состояние объекта может быть изменено только через Mutable ссылку. При попытке присвоить занчение поля через ReadOny ссылку произойдет ошибка компиляции:

\subsection{Объектная неизменяемость}

\subsection{Аннотации на методах}

\subsection{Исключение полей из абстрактного состояния объекта}

\subsection{Неизменяемые классы}

\subsection{Создание циклов неизменяемых объектов}




\section{Алгоритм вывода аннотаций}

\section{Сравнение с существующими подходами}


