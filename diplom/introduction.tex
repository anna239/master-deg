\chapter{Введение}

\section{Статический анализ кода}

\begin{Def}\label{static_program_analysis}
Статический анализ кода -- это анализ программного обеспечения, производимый без реального выполнения исследуемых программ.
\end{Def}

Статический анализ позволяет выявить многие виды ошибок еще до запуска программы, большинство из которых сложно искать и воспроизводить непосредственно во время работы приложения. В связи с этим активно развиваются различные инструменты, позволяющие статически доказывать отсутсвие в программах ошибок тех или иных видов.  

\textit{привести примеры тулов в исторической последовательности}

\section{Неизменяемость в контексте объектно-ориентированного языка}

В различных контекстах понятие неизменяемости может пониматься по-разному. В данной работе рассмотрено несколько видов неизменяемости:

\begin{Def}\label{immutabule_class}
Неизменяемый класс -- класс, все представители которого являются неизменяемыми. 
\end{Def}

Примером неизменяемого класса является, например, java.lang.String.

\begin{Def}\label{immutable_object}
Неизменяемый объект -- объект, который не может быть изменен, при не гарантируется, что другие представители того же самого класса могут быть изменены.
\end{Def}
\textit{добавить картинку}
Если в какая-либо система позволяет выражать данное свойство объекта, будем говорить, что в данной системе есть поддержка \textit{объектной неизменяемости}.

\begin{Def}\label{reference_immutability}
неизменяемая ссылка -- ссылка, которая не может быть использована для изменения объекта, на который она указывает (при этом объект может быть изменен через другую ссылку).
\end{Def} 

\textit{добавить кратинку}
Если какая-либо система позволяет выражать данное свойство объекта, будем говорить, что в данной системе есть поддержка \textit{ссылочной неизменяемости}. 

Нужно заметить, что данные понятия не являются чем-то искусственным по отношению к языкам программирования. Приведем примеры использования данных понятий в языке программирования Java.

Например, в документации к классу org.joda.time.Period написано: "Неизменяемый временной период..." \footnote{http://joda-time.sourceforge.net/apidocs/org/joda/time/Period.html}. Таким образом, класс org.joda.time.Period является неизменяемым классом. 
\textit{нужно ли приводить еще примеры}  

\input{prev_work}

\section{Постановка задачи}

Целью данной работы была разработка системы, позволяющей контролировать изменяемость объектов на этапе компиляции для языка Kotlin. 

К данной системе были предъявлены следующие требования:

\begin{itemize}
	\item Должна быть поддержана как объектная, так и ссылочная неизменяемость.
	
	\item Негобходима возможность исключать некоторые поля из абстрактного состояния объекта. 
	
	\item Данная система должна давать возможность создавать неизменяемые циклические структуры объектов.
	
	\item Необходимо иметь вохможность использовать уже существующий код.
\end{itemize}

В рамках данной работы решались следующие задачи:

\begin{itemize}

	\item Разработка системы аннотаций, позволяющей выражать неизменяемость объектов.
	
	\item Разработка алгоритма вывода аннотаций для существующего кода.

\end{itemize}



